\chapter{Under and overfitting}

In linear models, we can characterize forms of under and overfitting. For this chapter consider the following: \\ \ \\
\noindent
Model 1: $\bY = \bX_1 \bbeta_1 + \beps$ \\ \ \\ 
\noindent
Model 2: $\bY = \bX_1 \bbeta_1 + \bX_2 \bbeta_2 + \beps$ \\ \ \\
where the $\beps$ are assumed iid normals with variance $\sigma^2$. 
We further differentiate between the assumed model and the true model. 
If we assume Model 1 and Model 2 is true, we have underfit the model (omitted variables that were necessary). 
In contrast, if we assume Model 2 and Model 1 is true, we have overfit the model (included variables that were
unnecessary). 

\section{Impact of underfitting}
Consider underfitting the model. That is we errantly act as if Model 1 is true, but in fact model 2 is true.
Such a situation would arise if there were unmeasured or unknown confounders. Then consider the
bias of our estimate of $\bbeta1$. 

$$
E[\hat \bbeta_1] = E[(\bX_1^t \bX_1)^{-1} \bX_1^t \bY]
= (\bX_1^t \bX_1)^{-1} \bX_1^t (\bX_1 \bbeta_1 + \bX_2 \bbeta_2)
= \bbeta_1 + (\bX_1^t \bX_1)^{-1} \bX_1^t \bX_2 \bbeta_2.
$$

Thus, $(\bX_1^t \bX_1)^{-1} \bX_1^t \bX_2 \bbeta_2$ is the bias in estimating $\bbeta_1$. Notice
that there is no bias if $\bX_1^t \bX_2 = \bzero$. Consider the case where both design matrices
are centered. Then $\frac{1}{n-1}\bX_1^t \bX_2$ is the empirical 
variance/covariance matrix between the columns of $\bX_1$ and $\bX_2$. Thus, if our omitted
variables are uncorrelated with our included variables, then no bias exists. One way
to try to force this in practice is to randomize the levels of the variables in $\bX_1$.
Then, the empirical correlation will be low with high probability. This is very commonly
done when $\bX_1$ contains only a single treatment indicator.

Our theoretical standard errors for the $\hat \bbeta_1$ are still correct in that
$$
\Var(\hat \bbeta_1) = (\bX_1^t \bX_1)^{-1} \sigma^2.
$$
However, we still have to estimate $\sigma^2$. 

We can also see the  impact of underfitting on the bias of residual variance estimation.
\begin{eqnarray*}
E[(n- p_1) S^2] & = & E[\bY^t (\bI -  \bX_1 (\bX_1^t \bX_1)^{-1} \bX_1^t \bY] \\
& = &(\bX_1\bbeta_1 + \bX_2 \bbeta_2)^t\{\bI -  \bX_1 (\bX_1^t \bX_1)^{-1} \bX_1^t\} (\bX_1\bbeta_1 + \bX_2 \bbeta_2) \\
& + & \mathrm{trace}[\{\bI -  \bX_1 (\bX_1^t \bX_1)^{-1} \bX_1^t \} \sigma^2] \\
& = & \bbeta_2^t \bX_2^t \{\bI -  \bX_1 (\bX_1^t \bX_1)^{-1} \bX_1^t\} \bX_2 \bbeta_2 + (n - p_1) \sigma^2
\end{eqnarray*}
Therefore $S^2$ is biased upward. It makes sense that we would tend to overestimate the residual
variance if we've attributed to the error structure variation that is actually structured and
due to unmodeled systematic variation.

\section{Impact of overfitting}
Consider now fitting Model 2 when, in fact, Model 1 is true. There
is no bias in our estimate of $\bbeta_1$, since we have fit the correct model;
it's just $\bbeta_2 = \bzero$.  

\subsection{Variance inflation}
However, now consider $\mbox{Var}(\hat \bbeta_1) = \xtxinv \sigma^2$, where
$\bX = [\bX_1 ~ \bX_2]$. Recall, that the estimate for $\hat \bbeta_1$ can be
obtained by regression of $\be_{\bX_1 | \bX_{2}}$ on $\be_{\by | \bX_{2}}$. Thus,
$$
\hat \bbeta_1
= (\be_{\bX_1 | \bX_2}^t \be_{\bX_1 | \bX_2})^{-1} \be_{\bX_1 | \bX_2}^t \be_{\by | \bX_2}
$$
Let $\bH_{\bX_2}$ be the hat matrix for $\bX_2$. Thus,
\begin{eqnarray*}
\Var(\hat \bbeta_1) & = &
(\be_{\bX_1 | \bX_2}^t \be_{\bX_1 | \bX_2})^{-1} \be_{\bX_1 | \bX_2}^t (\bI - \bH_{\bX_2}) \be_{\bX_1 | \bX_2} (\be_{\bX_1 | \bX_2}^t \be_{\bX_1 | \bX_2})^{-1} \sigma^2 \\
& = & (\be_{\bX_1 | \bX_2}^t \be_{\bX_1 | \bX_2})^{-1} \sigma^2 + 
(\be_{\bX_1 | \bX_2}^t \be_{\bX_1 | \bX_2})^{-1} \be_{\bX_1 | \bX_2}^t \bH_{\bX_2} (\be_{\bX_1 | \bX_2}^t \be_{\bX_1 | \bX_2})^{-1} \sigma^2 \\
& =  &  (\be_{\bX_1 | \bX_2}^t \be_{\bX_1 | \bX_2})^{-1} \sigma^2.
\end{eqnarray*}
The latter term drops out since 
$$
\be_{\bX_1 | \bX_2}^t \bH_{\bX_2} = \bX_1^t (\bI - \bH_{X_2}) \bH_{\bX_2} = 0.
$$
Consider any linear contrast $\bq^t \bbeta_1$ then
\begin{eqnarray*}
\Var(\bq^t \hat \bbeta_1)& = &\bq^t (\be_{\bX_1 | \bX_2}^t \be_{\bX_1 | \bX_2})^{-1} \bq \sigma^2 \\
& = & \bq^t  (\bX_1^t\bX_1 - \bX_1 \bH_{\bX_2} \bX_1)^{-1} \bq \sigma^2 \\
& = &\bq^t \{(\bX_1^t\bX_1)^{-1} + \bW\}\bq \sigma^2
\end{eqnarray*}
where $\bW$ is a symmetric matrix. This latter identity can be obtained via the Woodbury theorem
\href{https://en.wikipedia.org/wiki/Woodbury_matrix_identity}{Wikipedia}.
Thus we can say that
$$
\Var(\bq^t \hat \bbeta_1) = \bq^t \{(\bX_1^t\bX_1)^{-1} + \bW\}\bq \sigma^2 
\geq \bq^t(\bX_1^t\bX_1)^{-1} \bq \sigma^2
$$
Therefore, the variance assuming Model 2 will always be greater than the variance assuming Model 1.
Note that at no point did we utilize which model was actually true. Thus we arrive at an
essential point, adding more regressors into a linear model necessarily increases the
standard error of the ones already included. This is called ``variation inflation''. The
estimated variances need not go up, since $\sigma^2$ will go down as we include variables. 
However, the central point is that one concern with including unnecessary regressors
is inflating a component of the standard error needlessly. 

Further note that $\sigma^2$ drops out in the ratio of the variances. We can thus
exactly calculate the percentage increase in variance caused by including regressors.
A particularly useful such summary is the variance inflation factor (VIF). 

\subsection{Variance inflation factors}
Assume that $\bX_1$ is a vector and that the intercept has been regressed out of 
both of $\bX_1$ and $\bX_2$. Recall from above that the variance for $\beta1$ assuming
Model 2 is (note $\beta_1$ is a scalar since we're assuming $\bX_1$ is a vector)
\begin{eqnarray*}
\Var(\beta_1) & = & (\be_{\bX_1 | \bX_2}^t \be_{\bX_1 | \bX_2})^{-1} \sigma^2.\\
& = & \frac{\sigma^2}{\bX_1^t (\bI - \bH_{\bX_2}) \bX_1} \\
& = & \frac{\sigma^2}{\bX_1^t \bX_1^t} \times \frac{\bX_1^t \bX_1}{\bX_1^t (\bI - \bH_{\bX_2}) \bX_1}
\end{eqnarray*}
Recall from partitioning sums of squares (remember that we've removed the intercept from both)
$$
\bX_1^t\bX_1 = \bX_1^t \bH_{\bX_2} \bX_1 + \bX_1^t (\bI - \bH_{\bX_2}) \bX_1
$$
and that $\frac{\bX_1^t \bH_{\bX_2} \bX_1}{\bX_1^t\bX_1}$ is the $R^2$ value for $\bX_1$
as an outcome and $\bX_2$ as a predictor. Let's call it $R^2_1$ so as not to confuse
it with the $R^2$ calcualted with $\bY$ as an outcome. Then we can write
$$
\Var(\beta_1) =  \frac{\sigma^2}{\bX_1^t \bX_1^t} \frac{1}{1 - R^2_1}.
$$
Note that $R^2 = 1$ if $\bX_2$ is orthogonal $\bX_1$. Thus,
$$
\frac{1}{1 - R^2_1}
$$
Is the relative increase in variability in estimating $\beta_1$
comparing the data as it is to the ideal case where $\bX_1$ is orthogonal to $\bX_2$. 
Similarly, since $\frac{\sigma^2}{\bX_1^t \bX_1^t}$ is the variance if $\bX_2$ is omitted
from the model. So $1 / (1 - R^2_1)$ is also the increase in the variance by adding the
other regressors in $\bX_2$.

This calculation can be performed for each regressor in turn. The $1 / (1 - R^2)$ value
for each regressor as an outcome with the remainder as predictors are the so-called
Variance Inflation Factors (VIFs). They give information about how much addition
variance is incurred by multicolinearity among the regressors.

\section{Variance under an overfit model}
If we fit Model 2, but Model 1 is correct, then our variance estimate is 
unbiased. We've fit the correct model, we just allowed the possibility that
$\bbeta_2$ was non-zero when it is exactly zero. Therefore $S^2$ is unbiased
for $\sigma^2$. Recall too that
$$
\frac{(n-p_1 - p_2)S^2_2}{\sigma^2} \sim \chi^2_{n-p_1 - p_2},
$$
where the subscript 2 on $S^2_2$ is used to denote the fitting where Model 2 was asssumed.
Similarly,
$$
 \frac{(n-p_1)S^2_1}{\sigma^2} \sim \chi^2_{n-p_1},
$$
where $S_1^2$ is the variance assuming Model 1 is true. Using the fact that the variance
of a Chi squared is twice the degrees of freedom, we get that
$$
\frac{\Var(S_2^2)}{\Var(S_1^2)} = \frac{(n - p_1)^2}{(n - p_1 - p_2)^2}.
$$
Thus, despite both estimates being unbiased, the variance of the estimated variance
under Model 2 is higher.


