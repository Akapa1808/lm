\chapter{Introduction}


Linear models is the cornerstone of statistical methodology.
Perhaps more than any other tool, advanced students of
statistics, biostatistics, machine learning, data science,
econometrics, etcetera should spend time learning the
finer grain details of this subject.

In this book, we give a treatment of advanced linear models.
It is advanced in the sense that it is of level that an
introductory PhD student in statistics or biostatistics
would see.

\section{Prerequisites}

Students will need a fair amount of mathematical prerequisites
before trying to undertake this class. First, is multivariate
calculus and linear algebra. Especially linear algebra, since
much of the early parts of linear models are direct applications
of linear algebra results applied in a statistical context.
In addition, some basic proof based mathematics is necessary
to follow the proofs.

We will also assume some basic mathematical statistics. The
courses [Mathematical Biostatistics Boot Camp 1](https://www.coursera.org/course/biostats)
and [Mathematical Biostatistics Boot Camp 2](https://www.coursera.org/course/biostats2)
by the author on Coursera would suffice. The [Statistical Inference](https://www.coursera.org/course/statinference) is a lower level
treatment that with some augmentated reading would also suffice.
There is a [Leanpub book](https://leanpub.com/LittleInferenceBook) for
this course as well.

Some basic regression would
be helpful too. The [Regression Models](https://www.coursera.org/course/regmods)
also by the author would suffice. Note that there is a
[Leanpub book](https://leanpub.com/regmods/) for this class. A
