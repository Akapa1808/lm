\chapter{Parameter estimability and penalties}

In this section we consider parameter estimability and penalties. 

\section{Estimability}

This section draws heavily from the wonderful book by \cite{searle2012linear}.

We define a linear combination of the slope parameters, $\bq^t \bbeta$, as being estimable
if it is equal to a linear combination of the expected value of $\bY$. In other words,
$\bq^t \bbeta$ is estimable if it is equal to $\bt^t E[\bY]$ for some value of $\bt$.

I find estimability most useful when $\bX$ is over-specified (not full rank). For example,
consider an ANOVA model
$$
Y_{ij} = \mu + \beta_i + \epsilon_{ij}.
$$
Verify for yourself that the $\bX$ matrix from this model is not full rank. 

Because $\bt^t E[\bY] = \bt^t \bX \bbeta$ for all possible $\bbeta$, $\bq = \bt^t \bX$ and
we obtain that estimable contrasts are necessarily linear combinations of the rows
of the design matrix. 

The most useful result in estimability is the invariance properties of estimable contrasts. Consider an not full rank design matrix.
Then any solution to the normal equations:
$$
\bX^t \bX \bbeta = \bX^t \bY
$$
will minimize the least squares criteria (or equivalently maximize the likelihood
under spherical Gaussian assumptions). (If you don't see this, verify it yourself using
the tools from the first few chapters.) Since $\bX$ is not full rank, this will
have infinite solutions. Let $\hat \bbeta$ and $\tilde \bbeta$ be any two such
solutions. For estimable quantities, $q^t \hat \bbeta = q^t \tilde \bbeta$. 
That is, the particular solution to the normal doesn't matter for estimable quantities. This should be
clear given the definition of estimability. Recall that least squares projects
onto the plane defined by linear combinations of the columns of $\bX$. The projection,
$\hat \bY$, is unique, while the particular linear combination is not in this case.

To discuss further. Suppose $\bq^t \hat \bbeta \neq \bq^t \tilde \bbeta$
for two solutions to the normal equations, $\hat \bbeta$ and $\tilde \bbeta$
and estimable $\bq^t \bbeta$. Then $\bt^t \bX \hat \bbeta \neq \bt^t \bX \tilde \bbeta$. Let
$\hat \bY$ be the projection of $\bY$ on the space of linear combinations of the columns of $\bX$.
However, since both are projections, $\hat \bY = \bX \hat \bbeta = \bX \tilde \bbeta$. 
Multiplying by $\bt^t$ then yields a contradiction.

\subsection{Why it's useful}

Consider the one way ANOVA setting again. 
$$
Y_{ij} = \mu + \beta_i + \epsilon_{ij}.
$$
For $i = 1, 2$, $j = 1, \ldots, J$. 
One can obtain parameter identifiability by setting $\beta_2 = 0$, $\beta_1 = 0$, $\mu = 0$ or $\beta_1 + \beta_2 = 0$ (or one of infinitely many other linear contrasts). 
These constraints don't change the column space of the $\bX$ matrix. (Thus the projection stays the same.)
Recall that $\hat y_{ij} = \bar y_i$. Estimable functions are linear combinations of $E[Y_{ij}] = \mu + \beta_i$. So, note that
$$
E[Y_{21}] - E[Y_{11}] = \beta_2 - \beta_1
$$
is estimable and it will always be estimated by $\bar y_2 - \bar y_1$. Thus, regardless of which linear constraints one points on the
model to achieve identifiability, the difference in the means will have the same estimate. 

This also gives us a way to go between estimates with different constraints without refitting the models. Since for two sets
of constraints we have:
$$
\bar y_i  = \hat \mu + \hat \beta_i = \tilde \mu + \tilde \beta_i,
$$
yielding a simple system of equations to convert between estimates with different constraints.


\section{Linear constraints}
Consider performing least squares under the full row rank linear constraints 
$$
\bK^t \bbeta = \bz.
$$
One could obtain these estimates using Lagrange multipliers
$$
||\by - \bX \bbeta||^2 + 2 \blambda^t (\bk^t \bbeta - \bZ)
= \by^t \by - 2\bbeta^2 \bX^t \by + \bbeta^t \bX^t \bX \bbeta + 2 \blambda^t (\bK^t \bbeta - \bz).
$$
Taking a derivative with respect to lambda yields
\begin{equation}
\label{eq:cls1}
2 (\bK^t \bbeta - \bz) = 0
\end{equation}
Taking a derivative with respect to $\bbeta$ we have:
$$
-2 \bX^t \by + 2 \bX^t \bX \bbeta + 2 \bK \blambda   = 0
$$
which has a solution in $\bbeta$ as
\begin{equation}
\label{eq:cls2}
\bbeta = \xtxinv (\bX^t \by  -   \bK \blambda) = \hat \bbeta - \xtxinv \bK \blambda,
\end{equation}
where $\hat \bbeta$ is the OLS (unconstrained) estimate.
Multiplying by $\bK^t$ and using \eqref{eq:cls1} we have that
$$
\bz = \bK^t \hat \bbeta - \bK^t \xtxinv \bK \blambda
$$
yielding a solution for $\blambda$ as
$$
\blambda = \{\bK^t \xtxinv \bK\}^{-1} (\bK^t \hat \bbeta - \bz). 
$$
Plugging this back into \eqref{eq:cls2} yields the solution:
$$
\bbeta = \hat \bbeta -  \xtxinv \bK \{\bK^t \xtxinv \bK\}^{-1} (\bK^t \hat \bbeta - \bz).
$$
Thus, one can fit constrained least squares estimates without actually refitting the model.
Notice, in particular, that if one where to multiply this estimate by $\bK^t$, the result would be
$\bz$.

\subsection{Likelihood ratio tests}
One can use this result to derive likelihood ratio tests of $H_0: \bK \bbeta = \bz$ versus
the general alternative. From the previous section, under the null hypothesis, the
estimate under the null hypothesis, 
$$\hat \bbeta_{H_0} = \hat \bbeta -  \xtxinv \bK \{\bK^t \xtxinv \bK\}^{-1} (\bK^t \hat \bbeta - \bz).$$
Of course, under the alternative, the estimate is $\hat \bbeta = \xtxinv \bX^t \bY$. In both
cases, the maximum likelihood variance estimate is $\frac{1}{n}||\bY - \bX \bbeta||^2$ with 
$\bbeta$ as the estimate under either the null or alternative hypothesis. Let $\hat \sigma^2_{H_0}$
and $\hat \sigma^2$ be the two estimates.

The likelihood ratio statistic is
$$
\frac{{\cal L}(\hat \bbeta_{H_0}, \hat \sigma^2_{H_0})}{{\cal L}(\hat \bbeta,\hat \sigma^2)}
= \left(\frac{\hat \sigma^2_{H_0}}{\hat \sigma^2} \right)^{-n/2}.
$$
This is monotonically equivalent to $n \hat \sigma^2 / n \hat \sigma^2_{H_0}$. However, we reject
if the null is less supported than the alternative, i.e. this statistic is small,
so we could equivalently reject if $n \hat \sigma^2_{H_0} / n \hat \sigma^2$ is large.
Further note
that 
\begin{eqnarray*}
n\hat \sigma^2_{H_0}
& = &||\bY - \bX \hat \bbeta_{H_0}||^2 \\
& = & ||\bY - \bX \hat \bbeta + \bX \hat \bbeta - \bX \hat \bbeta_{H_0}||^2 \\
& = & ||\bY - \bX \hat\bbeta||^2 + ||\bX \hat \bbeta - \bX \hat \bbeta_{H_0}||^2 \\
& = & n \hat \sigma^2 + ||\bX \hat \bbeta - \bX \hat \bbeta_{H_0}||^2
\end{eqnarray*}
Notationally, let $$SS_{reg} = ||\bX \hat \bbeta - \bX \hat \bbeta_{H_0}||^2
= ||\hat \bY - \hat \bY_{H_0}||^2$$ and
$SS_{res} = n \hat \sigma^2$. 
The note that the inverse of our likelihood ratio is monotonically equivalent to 
$\frac{SS_{reg}}{SS_{res}}$

However, $SS_{reg} / \sigma^2$ and $SS_{res} / \sigma^2$ are both independent 
Chi-squared random variables with degrees of freedom $Rank(\bK)$ and $n - p$
under the null. (Prove this for homework.) Thus, our likelihood ratio statistic can
exactly be converted into the $F$ statistic of section \ref{sec:ftest}. We leave
the demonstration that the two are identical as a homework exercise.

This line of thinking can be extended. Consider the sequence of hypotheses:
\begin{align*}
 & H_1 :\bK_1 \bbeta = \bz \\
 & H_2 :\bK_2 \bK_1 \bbeta = \bK_2 \bz \\
 & H_3 :\bK_3 \bK_2 \bK_1 \bbeta = \bK_3 \bK_2 \bz \\
 & \vdots 
\end{align*}
Each $\bK_i$ is assumed full row rank and of fewer rows than $\bK_{i-1}$. 
These hypotheses are nested with $H_1$ being the most restrictive, $H_2$ being the
second most, and so on. (Note, if $H_1$ holds then $H_2$ holds but not vice versa.)
Consider
testing $H_{1}$ (null) versus $H_{2}$ (alternative). Note that under our
general specification, discussing this problem will apply to testing $H_i$ versus $H_j$. 
Under the arguments
above, our likelihood ratio statistic will work out to be inversely equivalent to
the statistic: $n\hat \sigma^2_{H_1} / n\hat \sigma^2_{H_2}$. 


Further note that
\begin{eqnarray*}
n\hat \sigma^2_{H_1}
& = &||\bY - \hat \bY_{H_1}|| \\
& = & ||\bY - \hat \bY_{H_2}||^2  + ||\hat \bY_{H_2} - \hat \bY_{H_1}||^2
+ 2 (\bY - \hat \bY_{H_2})^t (\hat \bY_{H_2} - \hat \bY_{H_1}) \\
& = & ||\bY - \hat \bY_{H_2}||^2  + ||\hat \bY_{H_2} - \hat \bY_{H_1}||^2 \\
& = & SS_{RES}(H_2) + SS_{REG}(H_1 ~|~ H_2)
\end{eqnarray*}
Here the cross product term in the second line is zero by (tedious yet straightforward) algebra
and the facts that:
$\bK_2 \bK_1 \hat \bbeta_{H_1} = \bK_2 \bK_1 \hat \bbeta_{H_2} = \bK_2 \bz$ and $\be^t \bX = \bzero$. 

Thus, our likelihood ratio statistic is monotically equivalent to
$$SS_{REG}(H_1 ~|~ H_2) / SS_{RES}(H2).$$
Furthermore,
Using the developed methods in the class the numerator is Chi-Squared with
$Rank(\bK_1)$ degrees of freedom, while the denominator has $n - \{Rank(\bK_1) - Rank(\bK_2)\}$ 
degrees of freedom, and they are independent. Thus we can construct an F test for nested
linear hypotheses.

This process can be iterated, decomposing $SS_{RES}(H_2)$, so that:
$$
n\hat \sigma^2_{H_1}
= SS_{REG}(H_1 ~|~ H_2) + SS_{REG}(H_2 ~|~ H_3) + SS_{RES}(H_3)
$$
And it could be iterated again so that:
$$
n\hat \sigma^2_{H_1}
= SS_{REG}(H_1 ~|~ H_2) + SS_{REG}(H_2 ~|~ H_3) + \ldots SS_{RES}
$$
where $SS_{RES}$ is the residual sums of squares under no restrictions.
The sums of squares add so that, for example,
$$
SS_{REG}(H_1 ~|~ H_3) = SS_{REG}(H_1 ~|~ H_2) + SS_{REG}(H_2 ~|~ H_3)
$$
and
$$
SS_{RES}(H_3) = SS_{REG}(H_3 ~|~ H_4) + SS_{REG}(H_4 ~|~ H_4) + \ldots + SS_{RES}.
$$
Thus, one could test any subset of the nested hypotheses by appropriately adding the sums of
squares.

\subsection{Example use}
The most popular use of the general linear hypothesis is to consider nested hypotheses. 
That is, consider a linear model where $\bbeta^t = [\beta_0 ~ \beta_1 ~ \ldots ~\beta_p]$
so that the $\beta_i$ are ordered in decreasing scientific importance.
\begin{align*}
 & H_1 : \beta_1  = \beta_2  = \beta_3 = \ldots = \beta_p = 0 \\
 & H_2 :            \beta_2  = \beta_3 = \ldots = \beta_p = 0 \\
 & H_3 :                       \beta_3 = \ldots = \beta_p = 0 \\
 & \vdots \\
 & H_p : \beta_p = 0
\end{align*}
Then testing $H_1$ versus $H_2$ tests whether $\beta_1$ is zero under the assumption
that all of the remaining coefficients (excepting the intercept) are zero. 
Testing $H_2$ versus $H_5$ tests whether $\beta_2 = \beta_3 = \beta_4 = 0$
under the assumption that $\beta_5$ through $\beta_p$ are 0. 



\section{Ridge regression}

Consider quadratic constraints to least squares.
$$
||\bY - \bX \bbeta||^2 + \bbeta^t \bGamma \bbeta.
$$
In this case we consider instances where $\bX$ is not necessarily full rank. The
addition of the penalty is called ``Tikhonov regularization'' for the mathematician of
that name. The specific instance of this regularization in regression is called ridge
regression. The matrix $\bGamma$ is typically assumed known or set to $\gamma \bI$. 

Another way to envision ridge regression is to think in the terms of a posterior mode 
on a regression model. Specifically, $\bSigma^{-1} = \bGamma / \sigma^2$ and consider the model
where $\by ~|~ \bbeta \sim N(\bX \bbeta, \sigma^2 \bI)$ and $\bbeta \sim N(\bzero, \bSigma)$. 
Then one obtains the posterior for $\bbeta$ and $\sigma$ by multiplying the two densities. The
posterior mode would be obtained by minimizing minus twice the log of this product
$$
||\bY - \bX \bbeta||^2 / \sigma^2 + \bbeta^t \bGamma \bbeta / \sigma^2.
$$
which is equivalent to above in the terms of maximization for $\bbeta$.

We'll leave it as an exercise to obtain that the estimate actually obtained is
$$
\hat \bbeta_{ridge} = (\xtx + \bGamma)^{-1} \bX^t \bY.
$$
To see how this regularization helps with invertibility of $\xtx$, consider the case where
$\bGamma = \gamma \bI$. If $\gamma$ is very large then $\xtx + \gamma \bI$ is simply
small numbers added around an identity matrix, which is clearly invertible. 

Consider the case where $\bX$ is column centered and is of full column rank. 
Let $\bU \bD \bV^t$ be the SVD of $\bX$ where $\bU^t \bU = \bV ^t \bV = \bI$. 
Note then $\xtx = \bV \bD^2 \bV^t$ and $\xtxinv = \bV \bD^{-2} \bV^t$ so that
the ordinary least squares estimate satisfies
$$
\hat \bY = \bX \xtxinv \bX^t \bY = \bU \bD \bV^t \bV \bD^{-2} \bV^t \bV \bD \bU \bY = \bU \bU^t \bY.
$$
Consider now the fitted values under ridge regression with $\bGamma = \gamma \bI$:
\begin{align*}
\hat \bY_{ridge} 
& = \bX (\xtx + \gamma \bI)^{-1} \bX^t \bY \\
& = \bU \bD \bV^t (\bV \bD^2 \bV^t + \gamma \bI)^{-1} \bV \bD \bU^t \bY \\
& = \bU \bD \bV^t (\bV \bD^2 \bV^t + \gamma \bV \bV^t)^{-1} \bV \bD \bU^t \bY \\
& = \bU \bD \bV^t \{\bV (\bD^2 + \gamma \bI) \bV^t\}^{-1} \bV \bD \bU^t \bY \\
& = \bU \bD \bV^t \bV (\bD^2 + \gamma \bI)^{-1} \bV^t \bV \bD \bU^t \bY \\
& = \bU \bD (\bD^2 + \gamma \bI)^{-1} \bD \bU^t \bY\\
& = \bU \bW \bU^t \bY
\end{align*}
where the third line follows since $\bX$ is full column rank so that $\bV$ is $p\times p$ of full rank
and $\bV^{-1} = \bV^t$ so that $\bV^t \bV = \bV \bV^t = \bI$. Here $\bW$ is a diagonal matrix with elements
$$
\frac{D_i^2}{D_i^2 + \gamma}
$$
where $D_i^2$ are the eigenvalues. 

In the not full rank case, the same identity can be found, though it takes a bit more work. Now assume
that $\bX$ is of full row rank (i.e. that $n < p$ and there are no redundant subjects). Now note that
$\bV$ does not have an inverse, while $\bU$ does (and $\bU^{-1} = \bU^t$. Further note via
the Woodbury theorem (where $\theta = 1/\lambda)d$:
\begin{align*}
(\xtx + \gamma \bI)^{-1} & = 
\theta \bI - \theta^2 \bX^t (\bI + \theta \bX \bX^t)^{-1} \bX \\
& = \theta \bI - \theta^2 \bV \bD \bU^t  (\bU \bU^t + \theta \bU \bD^2 \bU^t)^{-1} \bU \bD \bV^t \\
& = \theta \bI - \theta^2 \bV \bD \bU^t \{\bU(\bI + \theta \bD^2) \bU^t)\}^{-1} \bU \bD \bV^t \\
& = \theta \bI - \theta^2 \bV \bD \bU^t \{\bU(\bI + \theta \bD^2)^{-1} \bU^t)\} \bU \bD \bV^t \\ 
& = \theta \bI - \theta^2 \bV \bD (\bI + \theta \bD^2)^{-1} \bD \bV^t \\
& = \theta \bI - \theta^2 \bV \tilde \bD \bV^t 
\end{align*}
where $\tilde \bD$ is diagonal with entries $D_i^2 / (1 + \theta D_i^2)$ where $D_i$ are the 
diagonal entries of $\bD$. Then:
\begin{align*}
\hat \bY_{Ridge} 
& = \bX (\xtx + \gamma \bI)^{-1} \bX^t \bY  \\
& = \bU \bD \bV^t (\theta \bI - \theta^2 \bV \tilde \bD \bV^t) \bV \bD \bU^t \bY \\
& = \bU \bD (\theta \bI - \theta^2 \tilde \bD)\bD \bU^t \bY \\
& = \bU \bW \bU^t \bY
\end{align*}
Thus we've covered the full row and column rank cases. (Omitting the instance where $\bX$ is neither
full row nor column rank.)

%Note that the inverse necessary for ridge regression
%can be obtained easily in the event that there are few rows but lots of columns. By the Woodbury %theorem (mentioned
%several times in this text), we have that
%$$
%(\xtx + \bGamma)^{-1} = \bGamma^{-1} - \bGamma^{-1} \bX^t (\bI + \bX \bGamma^{-1} \bX^t)^{-1} \bX %\bGamma^{-1}
%$$
%This identity is especially convenient when $\bGamma = \gamma \bI$. Note that the inverse necessary has the dimension of
%$\bX \bX^t$, which is the easier dimension to work with when $p$ is larger than $n$. 

\subsection{Coding example}



\section{Lasso regression}

The Lasso has been somewhat of a revolution of sorts in statistics and biostatistics
of late. The central idea of the lasso is to create a penalty that forces
coefficients to be zero. For centered $\bY$ and centered and scaled $\bX$, 
consider minimizing
$$
|| \bY - \bX \bbeta||^2
$$
subject to $\sum_{i=1}^p | \beta_i| < t$. The Lagrangian form of this minimization
can be written as minimizing
$$
|| \bY - \bX \bbeta||^2 + \lambda \sum_{i=1}^n | \beta_i|.
$$
Here $\blambda$ is a penalty parameter. As the Lasso constrains
$\sum_{i=1}^p | \beta_i| < t$, which has sharp corners on the axes,
it has a tendency to set parameters exactly to zero. Thus, it is 
thought of as doing model selection along with penalization. 
Moreover, the Lasso handles the $p > n$ problem. Finally, it's
a convex optimization problem, so that numerically solving 
for the Lasso is stable. We can more generally specify the parameter
as
$$
|| \bY - \bX \bbeta||^2 + \lambda \sum_{i=1}^n | \beta_i |^q.
$$
for $q > 0$. We obtain a case of ridge regression when $q=2$ and
the Lasso when $q=1$. Since $(\sum_{i=1}^n | \beta_i |^q)^{1/q}$ is a norm,
usually called the $l_q$ norm, the various forms of
regression are often calld $l_q$ regression. For example,
ridge regression could be called $l_2$ regression, the Lasso
${\cal L}_1$ regression and so on. We could write the penalized regression estimate as
$$
|| \bY - \bX \bbeta||^2 + \lambda ||\bbeta||_q^q
$$
where $||\cdot||_q$ is the $l_q$ norm.


You can visualize the parameters easily using Wolfram's alpha:
\href{http://www.wolframalpha.com/input/?i=abs\%28x1\%29+\%2B+abs\%28x2\%29+\%3D+1}{$|x_1| + |x_2|$},
\href{http://www.wolframalpha.com/input/?i=abs\%28x1\%29\%5E2+\%2B+abs\%28x2\%29\%5E2+\%3D+1}{$|x_1|^2 + |x_2|^2$},
\href{http://www.wolframalpha.com/input/?i=abs\%28x1\%29\%5E0.5+\%2B+abs\%28x2\%29\%5E0.5+\%3D+1}{$|x_1|^0.5 + |x_2|^0.5$},
and \href{http://www.wolframalpha.com/input/?i=plot+abs\%28x1\%29\%5E4+\%2B+abs\%28x2\%29\%5E4+\%3D+1}{$|x_1|^4 + |x_2|^4$}.
Notice that as $q$ tends to zero, it tends to all of the mass on the axes where as $q$ tends to infinity, it tends
to a square. The limit as $q$ tends to 0 is called the $l_0$ norm, which just penalizes the number
of non-zero coefficients. 

Just like with ridge regression, the Lasso has a Bayesian representation. Let
the prior on $\beta_i$ be iid from a Laplacian distribution with mean 0, which has
density $\frac{\theta}{2}\exp(- \theta |\beta_i|)$, and is denoted Lapplace$(0, \theta)$. 
Then, the Lasso estimate is
the posterior mean assuming $\by \sim N(\bX \bbeta, \sigma^2 I)$ and 
$\beta_i \sim_{iid} \mbox{Laplace}(0, \lambda / 2\sigma^2)$. Then minus twice 
the log of the posterior for $\bbeta$ conditioning on $\sigma$ is proportional to
$$
||\bY - \bX \bbeta||^2 + \lambda ||\bbeta||_{1}.
$$
Of course, the connection with Bayesian statistics is somewhat loose. While the Lasso is the
posterior mode conditioning on $\sigma$ under these assumptions, Bayesian analysis usually
take into account the full posterior, for which the parameter estimates are no longer sparse.







