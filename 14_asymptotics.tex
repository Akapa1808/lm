\chapter{Asymptotics}

\section{Convergence in probability}
A series of random variables, $Y_n$, is said the converge in probability to 
a constant $c$ if $P( | Y_n - c| > \epsilon) \rightarrow 0$ for any $\epsilon$. 
A standard result is that convergence in probability to the mean is implied if the
variance of random variable goes to zero (a consequence of \href{https://en.wikipedia.org/wiki/Chebyshev%27s_inequality}{Chebyshev's inequality}).
Specifically, let $Z_n = Y_n - \mu$ have mean $0$, variance 
$\sigma^2_n$
and distribution $F_n$
\begin{eqnarray*}
P( |Z_n | \geq  \epsilon) & = & 
\int_{|z_n | \geq \epsilon} dF_n (z_n) \\
& =  & 
\int_{z_n^2 / \epsilon^2 \geq 1} dF_n (z_n) \\
& \leq & \int_{z_n^2 / \epsilon^2 \geq 1} \frac{z_n^2}{\epsilon^2} dF_n (z_n)\\
& \leq & \int \frac{z_n^2}{\epsilon^2} dF_n (z_n)\\
& = & \sigma^2_n / \epsilon^2.
\end{eqnarray*}
Thus, according to our definition, $Z_n$ converges in probability 
to 0 (thus $Y_n$ converges in probability to $\mu$) if the
sequence of variances tends to zero. 

Consider now convergence in probability of our slope estimate
$$
\hat \bbeta_n = (\bX_n^t \bX_n)^{-1}\bX_n^t  \bY_n
$$
where subscripts have been added to denote the dependence on the sample size.
This estimator is unbiased for all $n$.
Under the assumption of iid errors, with a finite variance of $\bY_n$
of $\bI \sigma^2$, the variance of a linear contrast of $\hat \bbeta_n$ is
$$
\bq^t (\bX_n^t \bX_n)^{-1} \bq \sigma^2.
$$
Thus a sufficient condition for consistency of $\bq^t \hat \bbeta_n$ is 
for $\bq^t (\bX_n^t \bX_n)^{-1} \bq$ to converge to zero. Probably more
useful is if sample variance covariance associated with the $\bX_n$ converges,
then the estimate is consistent for all linear contrasts.

In the particular case for linear regression, recall that the variance of the slope
is 
$$
\frac{\sigma^2}{\sum_{i=1}^n (x_{i} - \bar{x})^2} =
\frac{\sigma^2}{(n-1) s^2_{x,n}}
$$
where $s^2_{x,n}$ is the sample variance of the $x$'s. Thus, as long as
$s_{x,n}^2$ converges, the estimate is consistent. Alternatively,
if the $X_i$ are bounded, then the estimate will converge. 



Consider now the case where $E[\bY_n] = \bX_n\bbeta$ but $\Var(\bY_n) = \bSigma_n$. Consider
a working covariance matrix, $\bW_n$, and the estimate
$$
\hat \bbeta(\bW_n) = (\bX^t_n \bW^{-1}_n \bX_n)^{-1} \bX^t_n \bW^{-1}_n \bY_n.
$$
The OLS estimate is the case where $\bW_n = \bI$. (Notice that the estimate is invariant to scale
changes in $\bW_n$.). Notice that $\hat \bbeta(\bW_n)$ is unbiased for all $n$ regardless of $\bW_n$.
The variance of $\hat \bbeta(\bW_n)$ is given by
$$
(\bX^t_n \bW^{-1}_n \bX_n)^{-1} \bX^t_n \bW^{-1}_n \bSigma_n  \bW^{-1}_n \bX_n (\bX^t_n \bW^{-1}_n \bX_n)^{-1}
$$
Thus, linear contrasts associated with $\hat \bbeta(\bW_n)$ will be consistent if both
$$
\frac{1}{n}(\bX^t_n \bW^{-1}_n \bX_n)
$$
and
$$
\frac{1}{n}(\bX^t_n \bW^{-1}_n \bSigma_n  \bW^{-1}_n \bX_n)
$$
converge. These are both weighted covariance matrices, weighting subjects via 
the working covariance matrix in the first case and $\bW^{-1}_n \bSigma_n  \bW^{-1}_n$ in the
latter. In the event that $\bW_n = \bI$ then the convergence of the former reduces to
convergence of the variance covariance matrix of the regression variables. However,
in more general cases, and the convergence of the latter weighted variance estimate,
cannot be given without further restrictions. 
One setting where convergence can be obtained is where $\bW_n$ and
$\bSigma_n$ have block diagonal structures as would be seen if one had repeated measurements
on subjects. 

In that case let $n$ be the number of subjects and $J$ be the number of observations within
subjects. Further let:
$\bX_n^t = [\bZ_1^t \ldots \bZ_n^t]$, $\bW_n = \bI_{n} \otimes \bW$ and $\bSigma_n = \bI_{n} \otimes \bSigma$ for 
$J \times p$ matrices $\bZ_i$ and $J\times J$
matrices $\bW$ and $\bSigma$. Think of each $\bZ_i$ as the covariates associated with the 
repeated measurements on subject $i$, $\bSigma$ is the within subject correlation and
$\bW$ is our working version of the within subject correlation. Then our two necessary convergent series are:
$$
\frac{1}{n}(\bX^t_n \bW^{-1}_n \bX_n) = \frac{1}{n} \sum_{i=1}^n \bZ_i^t \bW^{-1} \bZ_i
%=
% \mbox{Trace}\left(\bW^{-1}\frac{1}{n}\sum_{i=1}^n \bZ_i \bZ_i^t \right)
,
$$
and
$$
\frac{1}{n}(\bX^t_n \bW^{-1}_n \bSigma_n  \bW^{-1}_n \bX_n)=
\frac{1}{n} \sum_{i=1}^n \bZ_i^t \bW^{-1}\bSigma\bW^{-1} \bz_i = 
% \mbox{Trace}\left( \bW^{-1}\bSigma\bW^{-1} \frac{1}{n} \sum_{i=1}^n \bZ_i\bZ_i^t \right). 
$$
%Therefore, in both cases convergence of the 
%empirical variance covariance matrix of the $\bZ_i$ is enough
%to guarantee convergence.


\section{Normality}
Ascertaining convergence to normality is a bit more involved.
Fortunately, there's some convenient asymptotic theory to 
make life easier for us. Particularly, a version of the
Central Limit Theorem states that if $E[\epsilon_i] = 0$ and
$\Var(\epsilon_i) = \sigma2$ for $\epsilon_i$ iid and constants, $\bd^t
= [d_{n1} \ldots d_{nn}]$ then
$$
\frac{\sum_{i=1}^n d_{ni} \epsilon_i}{\sigma \sqrt{\sum_{i=1}^n d_{ni}^2 }}
=  \frac{ \bd_n ^ t \beps }{\sigma|| \bd_n ||} \rightarrow N(0, 1)
$$
if $\max d_{ni}^2 = o(\sum_{i=1}^n d_{ni}^2)$.

With this theorem, we have all that we need. Let
$\bY = \bX_n \bbeta + \beps$. Then note that
$$
\hat \bbeta_n  - \bbeta 
= (\bX_n^t \bX_n)^{-1} \bX_n^t \beps. 
$$
And so,
$$
\frac{\bq^t(\hat \bbeta_n  - \bbeta)}{\sigma \sqrt{\bq^t (\bX_n^t \bX_n)^{-1} \bq}}=
\frac{\bq^t (\bX_n^t \bX_n)^{-1} \bX_n^t \beps}%
{\sigma \sqrt{\bq^t (\bX_n^t \bX_n)^{-1} \bq}}
= \frac{\bd^t \beps}{\sigma ||\bd||}
$$
for $\bd = \bq^t (\bX_n^t \bX_n)^{-1}\bX_n$. Thus, our linear contrast
is $N(0,1)$, provided $\max \bq^t (\bX_n^t \bX_n)^{-1}\bX_n^t = o(\bq^t (\bX_n^t \bX_n)^{-1} \bq)$.
This will always be true if our $\bX_n$ matrix is bounded.

Consider now our case where $\hat \bbeta (\bw_n) = (\bX^t \bW_n \bX)^{-1} \bX^t \bY$. We
assume that $\bY^t = [\bY_1^t \ldots \bY_n^t]$, $\bX^t = [\bZ_1^t \ldots \bZ_n^t]$,
$\bW_n$ is a block matrix of $\bW$ as assumed before. For context consider repeated measurements
per subject. Let $\bY_i = \bZ_i \bbeta + \beps_i$ where $\Var(\beps_i) = \bSigma$. Then
relying on our earlier work:
\begin{align*}
\frac{\bq^t (\hat \bbeta(\bW_n) - \bbeta)}{\mbox{SD}\{\bq^t \hat \bbeta(\bW_n)\}}
& =  \frac{\bq^t(\sum_{i=1}^n \bZ_i^t \bW \bz_i)^{-1}\sum_{i=1}^n \bZ_i^t \bW \bSigma^{1/2} \bSigma^{-1/2} \beps_i}%
{\mbox{SD}\{\bq^t \hat \bbeta(\bW_n)\}} \\
& = \frac{\bq^t(\sum_{i=1}^n \bZ_i^t \bW \bz_i)^{-1}\sum_{i=1}^n \bZ_i^t \bW \bSigma^{1/2} \tilde \beps_i}%
{\mbox{SD}\{\bq^t \hat \bbeta(\bW_n)\}} \\
& = \frac{\sum_{i=1}^n \bd_{i} \tilde \beps_i}{\sqrt{\sum_{i=1}^n ||\bd_i ||^2}} \\
%& = \frac{\sum_{i=1}^n \frac{\bd_{i}}{||\bd_i||} \tilde \beps_i}{\sqrt{\sum_{i=1}^n ||\bd_i ||^2}} \\
& = \frac{\sum_{i=1}^n ||\bd_i|| \frac{\bd_i^t \tilde \beps_i}{||\bd_i||}}{\sqrt{\sum_{i=1}^n ||\bd_i ||^2}} \\
& = \frac{\sum_{i=1}^n ||\bd_i|| z_i}{\sqrt{\sum_{i=1}^n ||\bd_i ||^2}} 
\end{align*}
here $\tilde \beps_i$ is $N(0,\bI_{J})$, $z_i$ are iid with mean $0$ and variance $1$ and
$
\bd_i = \bq^t(\sum_{i=1}^n \bZ_i^t \bW \bz_i)^{-1} \bZ_i^t \bW \bSigma^{1/2}.
$
Thus we have reduced our statements down to the form of our generalized central limit theorem. 
Thus, we have shown that our estimates are asymptotically normal.

A final concern is that our statistic required $\bSigma$. However, a consequence of 
\href{https://en.wikipedia.org/wiki/Slutsky\%27s_theorem}{Slutsky's theorem} allows
us to replace it with any consistent estimate. Let:
$$
\be_i = \bY_i - (\bZ_i^t \bZ_i)^{-1} \bZ_i^t \bY_i 
$$
then 
$$
\frac{1}{n} \sum_{i=1}^n \be_i \be_i^t
$$
is a consistent estimate of $\bSigma$. A last concern is the issue of assuming equal
numbers of observations per subject. Generalizing this results in the same theory, just with more
notational difficulties. (So we'll just assume that it's taken care of). 
Thus we have a fully formed methodology for
performing inference on repeated measures data, where at no point did we presume
knowledge of $\bSigma$, or even a good estimate of it. This form of analysis 
was later generalized into Generalized Estimating Equations (GEEs). 





