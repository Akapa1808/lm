\section{The normal distribution}

%\href{https://www.youtube.com/watch?v=bI2YDQ8ABiA&list=PLpl-gQkQivXhdgUCdaUQcdb31CRe8Mm2y&index=44}{Watch this before beginning.}


\subsection{The univariate normal distribution}
A random variable $Z$ follows a standard normal distribution if its
density is 
$$
f(z) = \frac{1}{\sqrt{2\pi}} \exp(-z^2/2).
$$
We write the associated distribution function as $F$. 
A standard normal variate has mean 0 and variance 1. 
All odd numbered moments are 0. The non-standard
normal variate, say $X$, having mean $\mu$ 
and standard deviation $\sigma$ can be obtained
as $X = \mu + \sigma Z$. Conversely, 
$(X - \mu) / \sigma$ is standard normal if
$X$ is any non-standard normal. The
non-standard normal density is:
$$
f \left( \frac{x - \mu}{\sigma}\right) /\sigma
$$
with distribution function $F \left( \frac{x - \mu}{\sigma}\right)$.

\subsection{The multivariate normal distribution}

Suppose $Z_1, Z_2, \ldots Z_n$ are independent identically distributed (i.i.d.) standard normal random variables. The joint density of $\bfz = (Z_1, Z_2, \ldots Z_n)'$
is then given by
\begin{eqnarray*}
f_{\bfz}(\bfz) &=& \prod_{i=1}^p  \frac{1}{\sqrt{2\pi}} \exp(-z_i^2/2) \\
&=&  (2\pi)^{-p/2} \exp(-\bfz'\bfz/2) 
\end{eqnarray*}
This is the multivariate standard normal distribution for a 
random vector $\bz$ with  mean $\bzero$ and variance $\bI$.  We write this as $\bfz \sim N_p ({\bzero}, \bfI)$.

Non standard normal variates, say $\bx$, can be obtained as 
$\bx = \bmu + \bSigma^{1/2} \bz$ where $E[\bx] = \bmu$
and $\Var(\bfx) = \bSigma^{1/2}\bSigma^{1/2} = \bSigma$, which is assumed
to be positive definite. 
Conversely, one can go backwards with 
$\bz = \bSigma^{-1/2} (\bx - \bmu)$. The non-standard
multivariate normal distribution is given by
$$
(2\pi)^{-n/2} |\bSigma|^{-1/2} \exp\left\{ -\frac{1}{2}(\bx - \bmu)' \bSigma^{-1} (\bx - \bmu) \right\}.
$$
Commit this density to memory.  In this setting, we say that $\bfx \sim N_n (\bmu, \bSigma)$.

The normal distribution is nice to work with in that
all full row rank linear transformations of the normal are
also normal. That is, if $\ba + \bA \bx$ is normal if $\bA$ is
full row rank. Also, all conditional and submarginal distributions
of the multivariate normal are also normal. (We'll discuss the
conditional distribution more later.)

\subsection{Moment generating functions}

Given a vector $\bfmu$ and a positive
semidefinite matrix $\bfSigma$, $\bfy \sim N_n(\bfmu,\bfSigma)$ if the moment generating function (m.g.f.) of $\bfy$ is
$$M_\bfy(\bft) \equiv E[e^{\bft'\bfy}] = \exp\{
	\bfmu'\bft+{1\over 2}\bft'\bfSigma\bft \}.$$

\vb
Here are two important properties of moment generating functions:
\begin{itemize}
\item  If two random vectors have the same moment generating function, they have the same density.
\item Two random vectors are independent if and only if their joint moment generating function factors into the product of their two separate moment generating functions.
\end{itemize}

\subsection{Properties of the multivariate normal distribution}

Let  $\bfy \sim N_p(\bfmu,\Sigma)$, and let $\bf a$ be a $p \times 1$ vector, $\bfb$ a $k \times 1$ vector, and  $\bfC$ a $k \times p$ matrix with rank$=k \le p$, then,
\begin{itemize}
\item $x =  \bfa'\bfy  \sim N(\bfa'\bfmu,\bfa'\Sigma\bfa)$.

\vb
\item $\bfx =  \bfC\bfy + \bfb \sim N_p(\bfC\bfmu +\bfb,\bfC\Sigma\bfC')$.
\end{itemize}


\bexa
Let $\ \bfz=(Z_1,Z_2)' \sim N_2(\bf0,\bfI),\ $ and let $\bfA$ be the
linear transformation matrix
$$\bfA=\left(\begin{array}{rr}1/2&-1/2\\-1/2&1/2\end{array}\right).$$
Let $\bfy=(Y_1,Y_2)'$ be the linear transformation
$$\bfy=\bfA\bfz=
	\left(\begin{array}{c}(Z_1-Z_2)/2\\(Z_2-Z_1)/2\end{array}\right).$$
%Note that $\bfA=\bfA'=\bfA^2$ so $\bfA$ is a projection matrix
%(it projects from the plane onto the line defined by $Y_1+Y_2=0$.
Now $\ \bfy \sim N({\bf 0},\bfSigma)\ $ where
$\bfSigma=\bfA\bfA'$.
\eexa



\bstheo
\label{theo.mvn1}
Let $\ {\bf y}\sim N_n(\bfmu,\sigma^2 \bfI),\ $ and let $\bfT$ be an
orthogonal constant matrix.  Then $\ \bfT\bfy\sim
N_n(\bfT\bfmu,\sigma^2\bfI)$.
\etheo


Theorem \ref{theo.mvn1} says that mutually independent normal random
variables with common variance remain mutually independent with common
variance under orthogonal transformations.  Orthogonal matrices
correspond to rotations and reflections about the origin, i.e., they
preserve the vector length:
$$||\bfT\bfy||^2=(\bfT\bfy)'(\bfT\bfy) =
\bfy'\bfT'\bfT\bfy=\bfy'\bfy=||\bfy||^2.$$




\vb


\bstheo
If $\bfy \sim N_n(\bfmu,\bfSigma)$, then any $p \times 1$ subvector of $\bfy$ has a $p$-variate normal
distribution.
\estheo

It follows that if  $\bfy \sim N_n(\bfmu,\bfSigma)$ then $Y_i \sim N(\mu_i,\sigma^2_i)$ for $i=1, \ldots n$. Thus, joint normality implies marginal normality. The converse is not necessarily true.

\subsection{Singular normal}
%\href{https://www.youtube.com/watch?v=JGoX7lokhyc&index=45&list=PLpl-gQkQivXhdgUCdaUQcdb31CRe8Mm2y}{Watch this video before beginning.}


What happens if the $\bA$ in the previous section is not of full
row rank? Then $\Var(\bfX) = \bA\bSigma \bA'$ is not full rank.
There are redundant elements of the vector $\bX$ in that if
you know some of them, you know the remainder. 

An example is our residuals. The matrix $(\bI - \bfX (\bfX'\bfX)^{-1} \bfX)$ is not of full
rank (it's rank is $n-p$). For example, if we include an
intercept, the residuals must sum to 0. Know any $n-1$ of
them and you know the $n^{th}$. A contingency for this
is to define the singular normal distribution. A
singular normal random variable is any random variable
that can be written as $\bA \bz + \bb$ for a matrix $\bA$
and vector $\bb$ and standard normal vector $\bz$. 
As an example, consider the case where $\by \sim N(\bX \bbeta, \sigma^2 \bI)$.
Then the residuals, defined as 
$$\{\bI - \hatmat\} \by = \{\bI - \hatmat\}(\bX\bbeta + \frac{1}{\sigma} \bz)$$
are a linear transformation of iid normals. Thus the residuals are singular normal. 

The singular normal is such that all linear combinations and all submarginal and conditional distributions
are also singular normal. The singular normal doesn't necessarily
have a density function, because of the possibility of redundant
entries. For example, the vector $(Z ~ Z)$, where $Z$ is a standard normal,
doesn't have a joint density since the covariance matrix is $\bone_{2\times 2}$,
which isn't invertible.

The normal is the special case of the singular normal where the covariance
matrix is full rank. 


\subsection{Independence}

Let $\bf\sim N_n(\bfmu,\bfSigma)$ be partitioned as follows:
$$\bfy=\left(\begin{array}{c}\bfy_1 \\ \bfy_2 \end{array}\right),$$
where $\bfy_1$ is $p\times 1$ and $\bfy_2$ is $q\times 1$, ($p+q=n$).
Then, the mean and covariance matrix are correspondingly partitioned as
\begin{eqnarray*}
\bfmu=\left(\begin{array}{c}\bfmu_1 \\ \bfmu_2 \end{array}\right)
& \quad \mbox{and} \quad &
\bfSigma
=\left(\begin{array}{cc}
	\bfSigma_{11} & \bfSigma_{12}\\
	\bfSigma_{21} & \bfSigma_{22}
\end{array}\right)
=\left(\begin{array}{cc}
	\var(\bfY_1) & \cov(\bfY_1,\bfY_2) \\
	\cov(\bfY_2,\bfY_1) & \var(\bfY_2)
\end{array}\right).
\end{eqnarray*}

The marginal distributions are $\ \bfY_1 \sim
N_p(\bfmu_1,\bfSigma_{11})\ $ and $\ \bfY_2 \sim
N_q(\bfmu_2,\bfSigma_{22})$.

\vb
If $$\bfy=\left(\begin{array}{c}\bfy_1 \\ \bfy_2 \end{array}\right)$$ is $N_{p+q}(\bfmu, \Sigma)$, then 
$\bfy_1$ and $\bfy_2$ are independent if $\ \Sigma_{12} = {\bf \zero}$.
%Note, in this setting uncorrelated implies independent, that is  $\bfY_1$ and $\bfY_2$ are independent if and only if $\ \bfSigma_{12} = \bfSigma'_{21}=\zero$.
However, if $\ \bfY_1 \sim N_p(\bfmu_1,\bfSigma_{11})\ $ and $\ \bfY_2 \sim N_q(\bfmu_2,\bfSigma_{22})$, and $\ \bfSigma_{12} = \bfSigma'_{21}=\zero$, this does not necessarily mean that $\bY_1$ and $\bY_2$ are independent. We also need $\bfY$ to be jointly normal. It follows that if $\ \bfy \sim N_p(\bfmu, \Sigma)\ $, then any two individual variables $y_i$ and $y_j$ are independent if $\sigma_{ij}=0$.


\subsection{Conditional distributions}

%\href{https://www.youtube.com/watch?v=2VTf-XNmfAk&index=47&list=PLpl-gQkQivXhdgUCdaUQcdb31CRe8Mm2y}{Watch this video before beginning.}


%
%\bstheo
%If $\bfSigma$ is often of positive definite, then the conditional distribution
%of $\bfY_1$ given $\bfY_2$ is
%$$\bfY_1|\bfY_2=\bfy_2 \sim
%N_p(\bfmu_1 + \bfSigma_{12}\bfSigma_{22}^{-1}(\bfy_2 - \bfmu_2),
%\bfSigma_{11}-\bfSigma_{12}\bfSigma_{22}^{-1}\bfSigma_{21}).$$
%\etheo


The conditional distribution of a normal is often of interest.
Let $\bX = [\bX_1' ~ \bX_2']'$ be comprised of an $n_1 \times 1$
and $n_2 \times 1$ matrix where $n_1 + n_2  = n$. Assume that
$\bX \sim N(\bmu, \bSigma)$ where $\bmu = [\bmu_1' ~ \bmu_2']'$
and 
$$
\left[
\begin{array}{cc}
\bSigma_{11} & \bSigma_{12} \\
\bSigma_{12}' & \bSigma_{22}
\end{array}
\right].
$$
Consider now the conditional distribution of $\bX_1 ~|~ \bX_2$. 
A clever way to derive this (shown to me by a student in my class)
is as follows let $\bZ = \bX_1 + \bA \bX_2$
where $\bA = - \bSigma_{12}\bSigma_{22}^{-1}$. Then note that
the covariance between $\bX_2$ and $\bZ$ is zero (HW).

Thus the distribution of $\bZ ~|~ \bX_2$ is equal
to the distribution of $\bZ$ and that it is normally distributed
being a linear transformation of normal variates. Thus we know both
$$
E[\bZ ~|~ \bX_2 = \bx_2] = E[\bX_1 ~|~ \bX_2 = \bx_2] + \bA E[\bX_2 ~|~ \bX_2 = \bx_2]
= E[\bX_1 ~|~ \bX_2 = \bx_2] + \bA \bx_2
$$
and
$$
E[\bZ ~|~ \bX_2 = \bx_2] = E[\bZ] = \bmu_1 + \bA \bmu_2.
$$
Setting these equal we get that
$$
E[\bX_1 ~|~ \bX_2] = \bmu_1 + \bSigma_{12}\bSigma_{22}^{-1} (\bx_2 - \bmu_2).
$$
As a homework, using the same technique to derive the conditional variance
$$
\Var(\bZ ~|~ \bX_2 = \bx_2) = \bSigma_{11} - \bSigma_{12} \bSigma_{22}^{-1} \bSigma_{12}'.
$$

\subsection{An important example}
Consider the vector 
$(\bY ~ \bX')'$ where $\bY$ is $1\times 1$ and $\bX$ is $p\times 1$. Assume that 
the vector is normal with $E[\bY] = \mu_y$, $E[\bX] = \bmu_x$ and the variances
are $\sigma^2_y$ ($1\times 1$) and $\bSigma_x$ ($p\times p$) 
and covariance $\brho_{xy}$ ($p \times 1$).

Consider now predicting $\bY$ given $\bX = \bx$. Clearly a good estimate
for this would be $E[\bY ~|~ \bX = \bx]$. Our results
suggest that $\bY ~|~ \bX = \bx$ is normal with mean:
$$
\mu_y + \brho_{xy}' \bSigma_x^{-1} (\bx - \bmu_x)
= \mu_y - \bmu_x \bSigma_x^{-1}\brho_{xy} + \bx' \bSigma_x^{-1} \brho_{xy}
= \beta_0 + \bx' \bbeta
$$
where $\beta_0 = \mu_y - \bmu_x \bSigma_x^{-1}\brho_{xy}$ and $\beta = \bSigma_x^{-1} \brho_{xy}$. That is, the conditional mean in this case mirrors the
linear model. The slope is defined exactly as the inverse of the variance/covariance matrix
of the predictors times the cross correlations between the predictors and the response.
We discussed the empirical version of this in Section \ref{sec:lslin} where we saw that the empirical coefficients are the inverse of the empirical variance of the predictors times the empirical correlations between the predictors and response.  A similar mirroring occurs for the intercept as well. 

This correspondence simply says that empirical linear model estimates mirror the population parameters if both the predictors and response are jointly normal. It also yields
a motivation for the linear model in some cases where the joint normality of the
predictor and response is conceptually reasonable. Though we note that often such
joint normality is not reasonable, such as when the predictors are binary, even
though the linear model remains well justified. 

\subsection{A second important example}
Consider our partitioned variance matrix. 
$$
\bSigma = \left[
\begin{array}{cc}
\bSigma_{11} & \bSigma_{12} \\
\bSigma{12}' & \bSigma_{22}
\end{array}
\right].
$$
The upper diagonal element of $\bSigma^{-1}$ is given by $\bK_{11} = (\bSigma_{11} - \bSigma_{12} \bSigma_{22}^{-1} \bSigma_{12}')^{-1}$, which is the inverse of $\Var(\bX_1 ~|~ \bX_2)$. Suppose that $\bX_1 = (X_{11} ~ X_{12})'$. Then this result suggests that
$X_{11}$ is independent of $X_{12}$ given $\bX_2$ if the $(1,2)$ off diagonal element
of $\bSigma^{-1}$ is zero. To see this, recall that independence and absence of correlation are
equivalent in the multivariate normal. Hence, if $X_{11}$ is independent of $X_{12}$ given $\bX_2$ then $\bK_{11}^{-1}$ is diagonal, and so must $\bK_{11}$.
There's nothing in particular about the
first two positions, so we arrive at the following remarkable fact: whether or not
the off diagonal elements of $\bSigma^{-1}$ are zero determines the conditional
independence of those random variables given the remainder. This forms the
basis of so-called Gaussian graphical models. The graph defined by ascertaining
which elements of $\bSigma^{-1}$ are zero is called a conditional independence graph.


\subsection{Normal likelihood}

%\href{https://www.youtube.com/watch?v=HqlMCQwvjYw&list=PLpl-gQkQivXhdgUCdaUQcdb31CRe8Mm2y&index=46}{Watch this video before beginning.}

Let $\by \sim N(\bX \bbeta, \sigma^2 \bI)$ then note that 
minus twice the log-likelihood is:
$$
n\log(\sigma^2) + ||\by - \bX \bbeta||^2 / 2\sigma^2
$$
Holding $\sigma^2$ fixed we see that minmizing minus twice
the log likelihood (thus maximizing the likelihood) yields
the least squares solution:
$$
\hat \bbeta = \xtxinv \bX' \by.
$$
Since this doesn't depend on $\sigma$ it is the MLE. 
Taking derivatives and setting equal to zero we
see that 
$$
\hat \sigma^2 = ||\by - \bX \bbeta||^2 / n
$$
(i.e. the average of the squared residuals). We'll
find that there's a potentially preferable unbiased
estimate given by
$$
s^2 = ||\by - \bX \bbeta||^2 / (n - p).
$$

This model can be written in a  likelihood equivalent fashion
of  
$$
\by = \bX \bbeta + \beps
$$
where $\beps \sim N(0, \sigma^2 \bI)$. However, one
must be careful with specifying linear models this way.
For example, if one wants to simulate $Y = X + Z$ where $X$ and $Z$ are generated
independently, one can not equivalently simulate $X$
by generating $Y$ and $Z$ independently and taking $Z - Y$.
(Note $Y$ and $Z$ are correlated in the original simulation specification.)
Writing out the distributions explicitly removes all doubt. 
Thus the linear notation, especially when there are random effects, is sort
of lazy and imprecise (though everyone, your author included, uses it). 


Let's consider another case, suppose that $\by_1, \ldots, \by_n$ are iid $p$ vectors
$N(\bmu, \bSigma)$. Then, disregarding constants, minus twice the log likelihood is
$$
n \log|\bSigma| + \sum_{i=1}^n (\by_i - \bmu)' \bSigma^{-1} (\by_i - \bmu).
$$
Assume that $\bSigma$ is known, then using our derivative rules from earlier,
we can minimize this to obtain the MLE for $\bmu$
$$
\hat \mu = \bar \by 
$$
and the following for $\bSigma$
$$
\hat \bSigma = \frac{1}{n} \sum_{i=1}^n (\by_i - \bar \by) (\by_i -\bar \by)'
$$


Consider yet another case $\by \sim N(\bX\bbeta, \bSigma)$ with known $\bSigma$.
Minus twice the log-likelihood is:
$$
\log |\bSigma| + (\by - \bX \bbeta)' \bSigma^{-1} (\by - \bX \bbeta).
$$
Using our matrix rules we find that
$$
\hat \bbeta = (\bX' \bSigma^{-1} \bX)^{-1} \bX' \bSigma^{-1}\by. 
$$
This is the so-called generalized least squares estimate. 





\subsection{Bayes calculations}
We assume a slight familiarity of Bayesian calculations and inference for this section.
In a Bayesian analysis, one multiplies the likelihood times a prior distribution
on the parameters to obtain a posterior. The posterior distribution is then used for
inference. Let's go through a simple example.
Suppose that $\by ~|~ \mu \sim N(\bmu \bone_n, \sigma^2 I)$ and
$\mu ~|~ N(\mu_0, \tau^2)$ where $\by$ is $n \times 1$ and $\mu$ is a scalar. The
normal distribution placed on $\mu$ is called the "prior" and $\mu_0$ and $\tau^2$ 
are assumed to be known. For this example, let's assume that $\sigma^2$ is also
known. The goal is to calculate
$\mu ~|~ \by$, the posterior distribution. This is done by multiplying prior times
likelihood. Since, operarting generically,
$$
f(\mbox{Param} | \mbox{Data}) = \frac{f(\mbox{Param}, \mbox{Data})}{f(\mbox{Data})} 
\propto f(\mbox{Data} | \mbox{Param}) f(\mbox{Param}) = \mbox{Likelihood} \times \mbox{Prior}.
$$
Here, the proportional symbol, $\propto$, is with respect to the parameter.

Consider our problem, retaining only terms involving $\mu$ we have that minus twice the natural log of the
distribution of  $\mu ~|~ \by$ is given by
\begin{eqnarray*}
&   & -2 \log(f(\by ~|~ \mu)) - 2 \log(f(\mu)) \\
& = & ||\by - \mu \bone_n||^2 / \sigma^2  + (\mu - \mu_0)^2 / \tau^2 \\
& = & -2 \mu n \bar y / \sigma^2 + \mu^2 n / \sigma^2 + \mu^2 / \tau^2 - 2 \mu \mu_0 / \tau^2 \\
& = & -2 \mu \left(
\frac{\bar y}{\sigma^2 / n} + \frac{\mu_0}{\tau^2}\right)
+ \mu^2 \left(
\frac{1}{\sigma^2 / n} + \frac{1}{\tau^2}
\right)
\end{eqnarray*}
This is recognized as minus twice the log density of a normal distribution for $\mu$ with variance
of 
$$
\Var(\mu ~|~ \by) =
\left(
\frac{1}{\sigma^2 / n} + \frac{1}{\tau^2}
\right)^{-1}
=\frac{\tau^2 \sigma^2 / n}{\sigma^2 / n + \tau^2}
$$
and mean of
$$
E[\mu ~|~ \by] = \left(
\frac{1}{\sigma^2 / n} + \frac{1}{\tau^2}
\right)^{-1}
\left(
\frac{\bar y}{\sigma^2 / n} + \frac{\mu_0}{\tau^2}\right)
= p \bar y + (1 - p) \mu_0
$$
where 
$$
p = \frac{\tau^2}{\tau^2 + \sigma^2 /n }.
$$
Thus $E[\mu ~|~ \by]$ is a mixture of the empirical mean and the prior mean. How much 
the means are weighted depends on the ratio of the variance of the mean ($\sigma^2/n$)
and the prior variance ($\tau^2$). As we collect more data ($n \rightarrow \infty$), or if the
data is not noisy
($\sigma \rightarrow 0$) or we have a lot of prior uncertainty ($\tau \rightarrow \infty$) the empirical mean dominates. In contrast as we become more certain a priori
($\tau \rightarrow 0$) the prior mean dominates.


